\documentclass{article}
\usepackage[a4paper,bottom = 6.4 mm,left = 14.11 mm,right = 14.11 mm,top = 14.11 mm]{geometry}
\usepackage{graphicx}
\usepackage{amsmath}
\usepackage{array}
\usepackage{enumitem}
\usepackage{wrapfig}
\usepackage{microtype}
\usepackage{xcolor}
\usepackage{titlesec}
\usepackage{textcomp}
\usepackage[colorlinks=false]{hyperref}
\usepackage{verbatim}
\usepackage{titlesec}
\newcommand{\xfilll}[2][1ex]{
\dimen0=#2\advance\dimen0 by #1
\leaders\hrule height \dimen0 depth -#1\hfill}
\titleformat{\section}{\large\scshape\raggedright}{}{0em}{}
\renewcommand\labelitemi{\raisebox{0.4ex}{\tiny$\bullet$}}
\renewcommand{\labelitemii}{$\cdot$}
\pagenumbering{gobble}
\begin{document}


\vspace*{35 mm}
\noindent Pursuing a \textbf{Minor} degree in \textbf{ Artificial Intelligence} and \textbf{Data Science} from \textbf{C-MInDS, IIT Bombay}
\vspace{-15pt}
\noindent\section*{{\color{black!20!blue}\LARGE Scholastic Achievements}\xfilll[0pt]{0.5pt}}
\vspace{-12pt}

\begin{itemize}[itemsep = -1.4 mm, leftmargin=*]
    \item Achieved \textbf{99.81 Percentile} in {\bf JEE-Main} out of over 1 million candidates\hfill{\sl \small (2021)}
    \item  Secured {\bf All India Rank 1207} in {\bf JEE-Advanced} out of over 0.14 million candidates\hfill{\sl \small (2021)}
    \item Secured \textbf{AP grade} for excellent performance in \textbf{PH 108-Basics of Electricity \& Magnetism}\hfill{\sl \small (2022)}
    \item Secured a \textbf{Branch Change} to \textbf{Computer Science} department on the basis of academic performance\hfill{\sl \small (2022)}
    \item \textbf{Awarded fellowship} in the prestigious \textbf{KVPY (Kishore Vaigyanik Protsahan Yojna)} SX\hfill{\sl \small (2021)}


\end{itemize}
\vspace{-20pt}
\noindent\section*{\color{black!20!blue}\LARGE Key Projects\xfilll[0pt]{0.5pt}}
\vspace{-12pt}
\noindent\textbf{\large FastChat} \hfill{\sl \small Autumn 2022}\\
{\it Guide: Prof. Kavi Arya} $|$ {\it Ongoing Course Project : Software Systems Lab } \hfill{\it IIT Bombay}\\
\vspace{-17pt}
\begin{itemize}[itemsep = -1.4 mm, leftmargin=*]
    \item Developing a messaging software with \textbf{end-to-end encryption} by using \textbf{RSA+AES} to encode the messages and both \textbf{group chat} and \textbf{individual chat} support using \textbf{python socket library} and \textbf{PostgreSQL} database
    \item Implementing a \textbf{load balancer} with \textbf{least connect strategy} using \textbf{bash} to distribute load among multiple servers. Focusing on obtaining \textbf{high throughput} while using only \textbf{limited resources} dedicated for the servers
    \item Used \textbf{bash} scripts to simulate common messaging pattern and calculate \textbf{throughput} and \textbf{latency} of the system
\end{itemize}
\vspace{\baselineskip}
\vspace{-15pt}
\noindent\textbf{\large Lunar Lander using Deep Reinforcement Learning}\hfill{\sl \small Autumn 2022}\\
{\it Self Project}
\\\vspace{-17pt}
\begin{itemize}[itemsep = -1.4 mm, leftmargin=*]
    \item Implemented a \textbf{Deep Q learning} algorithm with \textbf{experience replay} using \textbf{TensorFlow} to train a lunar lander
    \item Used \textbf{Tensorflow Core} to define a \textbf{custom loss function} and a \textbf{custom training loop} using \textbf{GradientTape} to train the model. Used \textbf{epsilon greedy policy} to select the action with some amount of random decisions
    \item \textbf{Tuned and optimized model hyperparameters}, including learning rate, batch size, and number of episodes, epilon, gamma, number of timesteps to achieve the best results and solved the environment within 500 episodes
\end{itemize}
\vspace{\baselineskip}
\vspace{-15pt}
\noindent\textbf{\large Deep Learning and Neural Networks}\hfill{\sl \small Autumn 2022}\\
{\it Self Project}
\\\vspace{-17pt}
\begin{itemize}[itemsep = -1.4 mm, leftmargin=*]
    \item Made a \textbf{CNN} to classify images of handwritten digits using MNIST dataset and \textbf{TensorFlow, Keras and PyTorch} and compared their performance. Also made a \textbf{GUI} to draw digits using \textbf{python Tkinter} and classify them
    \item Made many different types of \textbf{CNNs} to classify various types of data such as \textbf{Traffic signs recogninition, Crack detection, Smile detection, Hand sign recogninition} using \textbf{PyTorch} and \textbf{Keras Sequential API}
    \item Successfully implemented \textbf{transfer learning} to train a \textbf{pretrained MobileNetV2} to classify images of \textbf{alpacas}
    \item Implemented \textbf{ResNet50's architecture} from scratch using \textbf{Keras API} and trained it on a hand sign dataset
\end{itemize}
\vspace{\baselineskip}
\vspace{-15pt}
\noindent\textbf{\large Machine Learning}\hfill{\sl \small Autumn 2022}\\
{\it Self Project}
\\\vspace{-17pt}
\begin{itemize}[itemsep = -1.4 mm, leftmargin=*]
    \item Learnt about the various machine learning algorithms such as \textbf{regressions, clustering, k-nearest neighbors, support vector machines and decision trees} and implemented them from scratch using numpy and pandas
    \item Proficiency in using \textbf{cross-validation and hyperparameter tuning} to optimize machine learning models
    \item Used \textbf{scikit-learn and XGBClassifer and XGBRegressor} to implement various types of classifiers and regressors to predict and classify various types of data such as \textbf{predicting house prices and classifying flower species}
\end{itemize}
\vspace{\baselineskip}
\vspace{-15pt}
\noindent\textbf{\large Rail Planner}\hfill{\sl \small Autumn 2022}\\
{\it Guide: Prof. Supratik Chakraborty} $|$ {\it Course Project : Data Structures and Algorithms Lab} \hfill{\it IIT Bombay}
\\\vspace{-17pt}
\begin{itemize}[itemsep = -1.4 mm, leftmargin=*]
    \item Developing a railway planner using algorithms such as \textbf{Merge Sort, KMP, Quicksort}, etc.
    \item Utilising Data Structures such as \textbf{linked lists, Binary Search Trees, AVL Trees, Hash tables, Tries}, etc.
\end{itemize}
% \vspace{\baselineskip}
% \vspace{-15pt}
% \noindent\textbf{\large Generating Representative Images from a Sample} \hfill{\sl \small Autumn 2022}\\
% {\it Guide: Prof. Suyash Awate} $|$ {\it Ongoing Course Project : Data Analysis and Interpretation } \hfill{\it IIT Bombay}
% \vspace{-3pt}
% \begin{itemize}[itemsep = -1.4 mm, leftmargin=*]
%     \item Used \textbf{MATLAB} to generate new
%           representative fruit images using \textbf{Principal Component Analysis (PCA)}
%     \item Used PCA to optimally reduce the
%           dimensionality of digits from the \textbf{MNIST Database} and reconstruct the image
% \end{itemize}
% \vspace{\baselineskip}
% \vspace{-15pt}
% \noindent\textbf{\large Multiplayer Tic-Tac-Toe} \hfill{\sl \small Autumn 2022}\\
% {\it Guide: Prof. Kavi Arya} $|$ {\it Course Project : Software Systems Lab } \hfill{\it IIT Bombay}\\
% \vspace{-20pt}
% \begin{itemize}[itemsep = -1.4 mm, leftmargin=*]
%     \item Used \textbf{Java Socket Programming} for \textbf{inter process communication} using the \textbf{peer-to-peer model}
%     \item Created the tic tac toe game using this model and handled various newtork and \textbf{IOStream exceptions}
% \end{itemize}
% \vspace{\baselineskip}
% \vspace{-15pt}
% \noindent\textbf{\large Bubble Trouble} \hfill{\sl \small Autumn 2021}\\
% {\it Guide: Prof. Parag Chaudhuri} $|$ {\it Course Project : Computer Programming and Utilization } \hfill{\it IIT Bombay}
% \vspace{-7pt}
% \begin{itemize}[itemsep = -1.4 mm, leftmargin=*]
%     \item Developed a video game using the \textbf{simplecpp graphics library} and object oriented programming in \textbf{C++} with a
%           physics simulation to model the motion of bubbles along with features such as timers, health bars, levels and scores
% \end{itemize}

\vspace{-20pt}
\noindent\section*{\color{black!20!blue}\LARGE Technical Skills\xfilll[0pt]{0.5pt}}
\vspace{-12pt}
\noindent\begin{tabular}{p{4.5cm} p{12.8cm}}
    \textbf{Programming Languages:}                        & C++, Python, MATLAB, Java, Bash, Solidity, Sed, AWK   \vspace{1pt}                                          \\
    \hline
    \vspace{1pt} \vspace{-8pt} \textbf{Software \& Tools:} & \vspace{1pt} \vspace{-8pt} Tensorflow, Pytorch, Keras, Scikit-learn, OpenCV, Seaborn, Git, \LaTeX{}, MySQL, \\ & NumPy, Pandas, Matplotlib, Doxygen, Sphinx \vspace{1pt} \\
    \hline
    \vspace{1pt} \vspace{-8pt} \textbf{Web Development:}   & \vspace{1pt} \vspace{-8pt} HTML, CSS, JavaScript, BootStrap                                                 \\
    %\textbf{Software} & Git, \LaTeX, AutoCAD
    %\textbf{Web Development} & Django, HTML, CSS, JavaScript, PHP, Bootstrap, Jquery, Android Studio  \\
    %   \textbf{Extra Courses} & Probability theory, Advance Network Security and Cryptography, Derivative Pricing
\end{tabular}
\vspace{-17pt}
\noindent\section*{\color{black!20!blue}\LARGE Extracurricular\xfilll[0pt]{0.5pt}}
\vspace{-12pt}
\begin{itemize}[itemsep = -1.4 mm, leftmargin=*]
    \item Successfully completed one year under \textbf{National Sports Organization(NSO)} in \textbf{Chess} at IIT Bombay\hfill{\sl \small (2022)}
    \item Pitched a \textbf{Business Model Canvas} for a startup in the health sector which entailed making online ambulance
          bookings, for the EnB Buzz competition conducted by the \textbf{Entrepreneurship cell of IIT Bombay}\hfill{\sl \small (2021)}
    \item Participated in a team of 3 and wrote a working script and successful submission in \textbf{Google Hashcode 2021}\hfill{\sl \small (2021)}
    \item Worked as team of 4 to make a remote controlled bot using ESP32 for XLR8 - an event of \textbf{ERC, IITB}\hfill{\sl \small (2022)}

\end{itemize}
\end{document}
