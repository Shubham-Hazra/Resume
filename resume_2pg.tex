\documentclass[a4paper,10pt]{article}
\usepackage[a4paper,bottom = 6.4 mm,left = 14.11 mm,right = 14.11 mm,top = 14.11 mm]{geometry}
\usepackage{graphicx}
\usepackage{amsmath}
\usepackage{array}
\usepackage{enumitem}
\usepackage{wrapfig}
\usepackage{microtype}
\usepackage{xcolor}
\usepackage{titlesec}
\usepackage{textcomp}
\usepackage[colorlinks=false]{hyperref}
\usepackage{verbatim}
\usepackage{titlesec}
\newcommand{\xfilll}[2][1ex]{
\dimen0=#2\advance\dimen0 by #1
\leaders\hrule height \dimen0 depth -#1\hfill}
\titleformat{\section}{\large\scshape\raggedright}{}{0em}{}
\renewcommand\labelitemi{\raisebox{0.4ex}{\tiny$\bullet$}}
\renewcommand{\labelitemii}{$\cdot$}
\pagenumbering{gobble}
\begin{document}


\vspace*{56.28 mm}
\noindent Pursuing a \textbf{Minor} degree in \textbf{Artificial Intelligence} and \textbf{Data Science} from \textbf{C-MInDS, IIT Bombay}
\vspace{-15pt}
\noindent\section*{{\color{black!20!blue}\LARGE Scholastic Achievements\xfilll[0pt]{0.5pt}}}
\vspace{-10pt}

\begin{itemize}[itemsep = -0.9 mm, leftmargin=*]
    \item Achieved \textbf{99.81 Percentile} in {\bf JEE-Main} out of over 1 million candidates\hfill{\sl \small (2021)}
    \item  Secured {\bf All India Rank 1207} in {\bf JEE-Advanced} out of over 0.14 million candidates\hfill{\sl \small (2021)}
    \item Secured {\bf AP(Advanced Performer) grade} for excellent performance in \textbf{PH 108-Basics of Electricity \& Magnetism}, awarded to 27 out of over 1300 students taking the course\hfill{\sl \small (2022)}
    \item One of the \textbf{17 out of 1400+} students to secure a \textbf{Change of Branch} to the department of \textbf{Computer Science and Engineering} owing to excellent academic performance in first year at IIT Bombay\hfill{\sl \small (2022)}
    \item Secured {\bf All India Rank 275} in the prestigious \textbf{KVPY (Kishore Vaigyanik Protsahan Yojna)} SX and awarded fellowship by the Department of Sciences, \textbf{Indian Institute of Science(IISC) Bangalore}\hfill{\sl \small (2021)}


\end{itemize}
\vspace{-20pt}
\noindent\section*{\color{black!20!blue}\LARGE Key Projects\xfilll[0pt]{0.5pt}}
\vspace{-10pt}
\noindent\textbf{\large FastChat} \hfill{\sl \small (Autumn 2022)}\\
{\it Guide: Prof. Kavi Arya} $|$ {\it Ongoing Course Project : Software Systems Lab } \hfill{\it IIT Bombay}\\
\vspace{-15pt}
\begin{itemize}[itemsep = -0.65 mm, leftmargin=*]
    \item Developing a messaging software by building a network of clients interacting via servers acting as mediators
    \item Focusing on obtaining \textbf{high throughput} while using only \textbf{limited resources} dedicated for the servers
    \item Ensuring \textbf{low latency} of individual message deliveries and \textbf{end-to-end encryption} between clients
    \item Using \textbf{python socket library} to develop the network, using \textbf{open source libraries} for authentication and communication, \textbf{PostgreSQL} database to store the data and \textbf{bash} for scripting and collecting results
\end{itemize}
\vspace{\baselineskip}
\vspace{-15pt}
\noindent\textbf{\large Lunar Lander using Deep Reinforcement Learning}\hfill{\sl \small (Autumn 2022)}\\
{\it Self Project}
\\\vspace{-15pt}
\begin{itemize}[itemsep = -0.65 mm, leftmargin=*]
    \item Used \textbf{Pygame} and \textbf{OpenAI's Gym} to train a lunar lander game \textbf{Deep Q-Learning with Experience Replay}. Used the \textbf{Sequential API} of the Keras library to define the \textbf{Q-network} and the \textbf{target Q-network}
    \item Used \textbf{Tensorflow Core} to define a \textbf{custom loss function} and a \textbf{custom training loop} using \textbf{GradientTape} to train the model. Used \textbf{epsilon greedy policy} to select the action with some amount of random decisions
    \item Utilized a \textbf{deque} for storing the experience buffer and used \textbf{experience replay} and \textbf{soft update} of the Q targets to stabilize the learning process and improve the model's convergence towards an optimal solution
    \item \textbf{Tuned and optimized model hyperparameters}, including learning rate, batch size, and number of episodes, epilon, gamma, number of timesteps to achieve the best results and solved the environment within 500 episodes
\end{itemize}
\vspace{\baselineskip}
\vspace{-15pt}
\noindent\textbf{\large Deep Learning and Neural Networks}\hfill{\sl \small (Autumn 2022)}\\
{\it Self Project}
\\\vspace{-15pt}
\begin{itemize}[itemsep = -0.65 mm, leftmargin=*]
    \item Made a \textbf{convolutional neural network} to classify images of handwritten digits using \textbf{MNIST} dataset and also made a \textbf{GUI} to draw digits using \textbf{python Tkinter} and classify them using the model. Made three different models using \textbf{TensorFlow Core, Keras Functional API and PyTorch} and compared their performance
    \item Made many different types of \textbf{CNNs} to classify various types of data such as \textbf{Traffic signs recogninition, Crack detection, Smile detection, Hand sign recogninition} using \textbf{PyTorch} and \textbf{Keras Sequential API}
    \item Successfully implemented \textbf{transfer learning} to train a \textbf{pretrained MobileNetV2} to classify images of \textbf{alpacas} and used \textbf{data augmentation} like random rotation and flipping resulting in a highly accurate model
    \item Implemented \textbf{ResNet50's architecture} from scratch using \textbf{Keras Functional API} and trained it on a hand sign dataset to classify images of 6 different classes. Compared its performance to a \textbf{pretrained ResNet50 model}
\end{itemize}
\vspace{\baselineskip}
\vspace{-15pt}
\noindent\textbf{\large Generating Representative Images from a Sample} \hfill{\sl \small (Autumn 2022)}\\
{\it Guide: Prof. Suyash Awate} $|$ {\it Ongoing Course Project : Data Analysis and Interpretation } \hfill{\it IIT Bombay}
\vspace{-3pt}
\begin{itemize}[itemsep = -0.65 mm, leftmargin=*]
    \item Used \textbf{MATLAB} to implement a Monte Carlo simulation of a given distribution and plotted the PDF and CDF
    \item Used \textbf{MATLAB} to use a data set of images of various fruits and sampled random images to generate new
          representative fruit images using \textbf{Principal Component Analysis (PCA)}. Also used PCA to analyse images of handwritten digits from the \textbf{MNIST Database} and optimally \textbf{reduce the
              dimensionality} and \textbf{reconstruct} the images
\end{itemize}

\pagebreak
\vspace{\baselineskip}
\vspace{-20pt}
\noindent\textbf{\large Machine Learning}\hfill{\sl \small (Autumn 2022)}\\
{\it Self Project}
\\\vspace{-15pt}
\begin{itemize}[itemsep = -0.65 mm, leftmargin=*]
    \item Experience in machine learning frameworks such as \textbf{scikit-learn, XGBoost, PyTorch, TensorFlow, and Keras}.
    \item Using Python packages such as \textbf{numpy, pandas, matplotlib and seaborn} for data manipulation and analysis. Learnt about \textbf{feature engineering} and feature selection techniques to improve the performance of the models
    \item Learnt about the various machine learning algorithms such as \textbf{regressions, clustering, k-nearest neighbors, support vector machines and decision trees} and implemented them from scratch using numpy and pandas
    \item Proficiency in using \textbf{cross-validation and hyperparameter tuning} to optimize machine learning models
    \item Used \textbf{scikit-learn and XGBClassifer and XGBRegressor} to implement various types of classifiers and regressors to predict and classify various types of data such as \textbf{predicting house prices and classifying flower species}
\end{itemize}
\vspace{\baselineskip}
\vspace{-15pt}
\noindent\textbf{\large Rail Planner}\hfill{\sl \small (Autumn 2022)}\\
{\it Guide: Prof. Supratik Chakraborty} $|$ {\it Course Project : Data Structures and Algorithms Lab} \hfill{\it IIT Bombay}
\\\vspace{-15pt}
\begin{itemize}[itemsep = -0.65 mm, leftmargin=*]
    \item Designed a simplified vesrion of a railway planner using various data structures and analyzed the space \& time complexity and the efficiency to demonstrate the \textbf{properties of different data structures in C++}
    \item Stored trains as a dictionary using \textbf{Hash Tables} and devised algorithms for fastest possible journies
    \item Used \textbf{BSTs and then AVL trees} for quick searching using the journey codes and used \textbf{Tries} to implement the autocompletion feature while searching for station names and added a feature to accept reviews for journies
    \item Used \textbf{Quicksort} to order trains by day and time, implemented the \textbf{KMP-string matching algorithm} for allowing review searches by using keywords and implemented \textbf{Heaps} to allow filtering the reviews by their rating
\end{itemize}
% \pagebreak
% \vspace{\baselineskip}
% \vspace{-15pt}
% \noindent\textbf{\large Monte Carlo Analysis of Statistical Theorems} \hfill{\sl \small (Autumn 2022)}\\
% {\it Guide: Prof. Suyash Awate} $|$ {\it Course Project : Data Analysis and Interpretation } \hfill{\it IIT Bombay}
% \vspace{-3pt}
% \begin{itemize}[itemsep = -0.65 mm, leftmargin=*]
%     \item Used \textbf{MATLAB} to implement a Monte Carlo simulation of a given Probability distribution
%     \item Plotted the probability and cumulative distribution functions of various distributions and empirically verified various statistical theorems such as the law of large numbers, Poison thinning and the
%           Gaussian nature of the Random Walk
% \end{itemize}
% \vspace{\baselineskip}
% \vspace{-12pt}
% \noindent\textbf{\large Text File Editors}\hfill{\sl \small (Autumn 2022)}\\
% {\it Guide: Prof. Kavi Arya} $|$ {\it Course Project : Software Systems Lab } \hfill{\it IIT Bombay}
% \\\vspace{-15pt}
% \begin{itemize}[itemsep = -0.65 mm, leftmargin=*]
%     \item Developed an analog to the Linux Command Line utility \textbf{wc command} using the \textbf{awk programming language}
%           that counts the number of characters, words and lines in a text file and also accepts flags similar to wc command
%     \item Developed a program to check for valid email addresses using \textbf{sed} with pattern matching using \textbf{regular expressions}
%     \item Implemented a \textbf{csv file editor} that formats columns based on customisable properties such as date, time and name
%     \item Developed a program which changes the base of the number to a different given base using \textbf{bash scripting and awk}
%     \item Developed a program to \textbf{encrypt} a piece of text when the words to encrypt and their corresponding cipher is given
% \end{itemize}
\vspace{\baselineskip}
\vspace{-15pt}
\noindent\textbf{\large Multiplayer Tic-Tac-Toe} \hfill{\sl \small (Autumn 2022)}\\
{\it Guide: Prof. Kavi Arya} $|$ {\it Course Project : Software Systems Lab } \hfill{\it IIT Bombay}\\
\vspace{-15pt}
\begin{itemize}[itemsep = -0.65 mm, leftmargin=*]
    \item Used \textbf{Java Socket Programming} for \textbf{inter process communication} using the \textbf{peer-to-peer model}
    \item Created the tic tac toe game using this model and handled various newtork and \textbf{IOStream exceptions}
\end{itemize}
\vspace{\baselineskip}
\vspace{-15pt}
\noindent\textbf{\large Personal Website}\hfill{\sl \small (Autumn 2022)}\\
{\it Guide: Prof. Kavi Arya} $|$ {\it Course Project : Software Systems Lab } \hfill{\it IIT Bombay}
\\\vspace{-15pt}
\begin{itemize}[itemsep = -0.65 mm, leftmargin=*]
    \item Made a personal website to be hosted on the CSE department server using \textbf{HTML and CSS and JavaScript}
    \item Added various advanced \textbf{CSS} features animations, transitions, static scroll images, modals, checkboxes and slideshows
    \item Used \textbf{JavaScript} to make the website interactive, gauge user-choices and render web-pages accordingly and deployed the website on an SSH server; used \textbf{BootStrap} to impelement standard navigation bars, footers and other features
\end{itemize}
\vspace{\baselineskip}
\vspace{-15pt}
\noindent\textbf{\large Bubble Trouble} \hfill{\sl \small Autumn 2021}\\
{\it Guide: Prof. Parag Chaudhuri} $|$ {\it Course Project : Computer Programming and Utilization } \hfill{\it IIT Bombay}
\vspace{-2pt}
\begin{itemize}[itemsep = -0.65 mm, leftmargin=*]
    \item Developed a video game using the \textbf{simplecpp graphics library} and object oriented programming in \textbf{C++} with a
          physics simulation to model the motion of bubbles along with features such as timers, health bars, levels and scores
\end{itemize}
% \vspace{\baselineskip}
% \vspace{-10pt}
% \noindent\textbf{\large Bubble Trouble} \hfill{\sl \small (Spring 2021)}\\
% {\it Guide: Prof. Parag Chaudhuri} $|$ {\it Course Project : Computer Programming and Utilization } \hfill{\it IIT Bombay}
% \vspace{-3pt}
% \begin{itemize}[itemsep = -0.65 mm, leftmargin=*]
%     \item Designed an interactive single player retro style game which impelements a bubble shooter to shoot random floating bubbles on the screen to demonstrate the \textbf{Object Oriented Paradigm in C++}
%     \item Implemented event-handling using \textbf{XEvent} object extensively used the \textbf{C++ STL} and the Simplecpp library that was developed in-house by the institute to add the various features of the game
%     \item Handled various events, assigning multiple responses by the game and designed the game for many levels of difficulty
% \end{itemize}

\vspace{-15pt}
\noindent\section*{\color{black!20!blue}\LARGE Technical Skills\xfilll[0pt]{0.5pt}}
\vspace{-9pt}
\noindent\begin{tabular}{p{4.5cm} p{13.2cm}}
    \textbf{Programming Languages}                        & C++, C, Python, MATLAB, Java, Bash, Solidity, Sed, AWK     \vspace{2pt}                                     \\
    \hline
    \vspace{2pt} \vspace{-7pt} \textbf{Software \& Tools} & \vspace{2pt} \vspace{-7pt} Tensorflow, Pytorch, Keras, Scikit-learn, OpenCV, Seaborn, Git, \LaTeX{}, MySQL, \\ & NumPy, Pandas, Matplotlib, Doxygen, Sphinx  \vspace{2pt}\\
    \hline
    \vspace{2pt} \vspace{-7pt} \textbf{Web Development}   & \vspace{2pt} \vspace{-7pt} HTML, CSS, JavaScript, BootStrap                                                 \\
\end{tabular}
\vspace{-12pt}

%\vspace{-17pt}

\noindent\section*{\color{black!20!blue}\LARGE Courses Undertaken\xfilll[0pt]{0.5pt}}
\vspace{-8pt}
\noindent\begin{tabular}{m{36mm} m{13.2cm}}
    \textbf{Mathematics}  \vspace{2pt}                   & Calculus, Linear Algebra, Differential Equations, Optimization Models   \vspace{2pt}                                                                                \\
    \hline
    \vspace{3pt} \vspace{-1pt} \textbf{Computer Science} & \vspace{3pt} \vspace{-1pt}  Data Structures and Algorithms\textsuperscript{\#}, Discrete Structures, Data Analysis and Interpretation, Software Systems Laboratory,
    Computer Networks*\textsuperscript{\#},, Digital Logic Design*\textsuperscript{\#}, Design and Analysis of Algorithms*,
    Logic for Computer Science*, Introduction to Blockchains Cryptocurrencies and Smart Contracts*, Computer Vision*\vspace{2pt}                                                                                               \\
    \hline
    \vspace{3pt} \vspace{-1pt} \textbf{Miscellaneous}    & \vspace{3pt} \vspace{-1pt} Game Theory and Decision Analysis*, Introduction to Electric and Electronic Circuits, Quantum Physics and Application,
    Basics of Electricity and Magnetism, Engineering Graphics and Drawing, Organic and
    Inorganic Chemistry, Physical Chemistry, Biology                                                                                                                                                                           \\                                                                                                                    \\
    % \textbf{Online Courses}   & Machine Learning
\end{tabular}
\vspace{-10pt}
\begin{flushright}
    \sl\small(* to be completed by April 2023)\\ \sl \small (\# Theory + Lab) \\
\end{flushright}
\vspace{-15pt}
\noindent\section*{\color{black!20!blue}\LARGE Extracurricular\xfilll[0pt]{0.5pt}}
\vspace{-7pt}
\begin{itemize}[itemsep = -0.65 mm, leftmargin=*]
    \item Successfully completed one year under \textbf{National Sports Organization(NSO)} in \textbf{Chess} at IIT Bombay\hfill{\sl \small (2022)}
    \item Pitched a \textbf{Business Model Canvas} for a startup in the health sector which entailed making online ambulance
          bookings, for the EnB Buzz competition conducted by the \textbf{Entrepreneurship cell of IIT Bombay}\hfill{\sl \small (2021)}
    \item Participated in a team of 3 and wrote a working script and successful submission in \textbf{Google Hashcode 2021}\hfill{\sl \small (2021)}
    \item Worked in a team of 4 to make an ESP32 \textbf{WiFi-controlled} bot for XLR8 conducted by \textbf{ERC, IITB}\hfill{\sl \small (2022)}

\end{itemize}
\end{document}