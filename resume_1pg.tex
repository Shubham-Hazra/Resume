\documentclass[a4paper,10pt]{article}
\usepackage[a4paper,bottom = 6.4 mm,left = 14.11 mm,right = 14.11 mm,top = 14.11 mm]{geometry}
\usepackage{graphicx}
\usepackage{amsmath}
\usepackage{array}
\usepackage{enumitem}
\usepackage{wrapfig}
\usepackage{microtype}
\usepackage{xcolor}
\usepackage{titlesec}
\usepackage{textcomp}
\usepackage[colorlinks=false]{hyperref}
\usepackage{verbatim}
\usepackage{titlesec}
\newcommand{\xfilll}[2][1ex]{
\dimen0=#2\advance\dimen0 by #1
\leaders\hrule height \dimen0 depth -#1\hfill}
\titleformat{\section}{\large\scshape\raggedright}{}{0em}{}
\renewcommand\labelitemi{\raisebox{0.4ex}{\tiny$\bullet$}}
\renewcommand{\labelitemii}{$\cdot$}
\pagenumbering{gobble}
\definecolor{mycolor}{RGB}{51, 102, 204}
\setlength{\parindent}{0pt}
\begin{document}


\vspace*{35 mm}
\noindent Pursuing a \textbf{Minor} degree in \textbf{Artificial Intelligence} and \textbf{Data Science} from \textbf{C-MInDS, IIT Bombay}
\vspace{-17pt}
\noindent\section*{{\color{mycolor}\LARGE Scholastic Achievements}\xfilll[0pt]{0.5pt}}
\vspace{-12pt}

\begin{itemize}[itemsep = -1.4 mm, leftmargin=*]
    \item Achieved \textbf{99.81 Percentile} in {\bf JEE-Main} out of over 1 million candidates\hfill{\sl \small (2021)}
    \item Achieved {\bf 99.14 Percentile} in {\bf JEE-Advanced} out of over 0.14 million candidates\hfill{\sl \small (2021)}
    \item Achieved \textbf{AP grade} for excellent performance in \textbf{PH 108-Basics of Electricity \& Magnetism}\hfill{\sl \small (2022)}
    \item Secured a \textbf{Branch Change} to \textbf{Computer Science} department on the basis of academic performance\hfill{\sl \small (2022)}
    \item Secured {\bf AIR 275} in the prestigious \textbf{KVPY SX} and awarded fellowship by \textbf{IISc Bangalore}\hfill{\sl \small (2021)}


\end{itemize}

\vspace{-20pt}
\noindent\section*{{\color{mycolor}\LARGE Work Experience\xfilll[0pt]{0.5pt}}}
\vspace{-13pt}
\noindent\textbf{\large Applied AI Researcher at Brance Technologies}\hfill{\sl \small (Summer 2023)}\\
\vspace{-17pt}
\begin{itemize}[itemsep = -1.4 mm, leftmargin=*]
    \item Developed high-performance chatbot systems using \textbf{vector embeddings }and \textbf{Large Language Models (LLMs)}
    \item Utilized \textbf{Haystack, FAISS, vectorDBs}, and \textbf{Hugging Face models} for indexing, retrieval and ranking of data
    \item Employed \textbf{Locality-Sensitive Hashing (LSH)} for caching queries with \textbf{semantic search}, optimizing data retrieval 
    \item Leveraged \textbf{Nginx, FastAPI} and \textbf{async calls}, on an \textbf{AWS EC2} for seamless communication and reduced latency
\end{itemize}

\vspace{-20pt}
\noindent\section*{\color{mycolor}\LARGE Key Projects\xfilll[0pt]{0.5pt}}

\vspace{-10pt}
\noindent\textbf{\large Stable Diffusion from Scratch}\hfill{\sl \small (Summer 2023)}\\
{\it Self Project}
\\\vspace{-17pt}
\begin{itemize}[itemsep = -1.4 mm, leftmargin=*]
    \item Developed and implemented each component of a \textbf{Stable Diffusion} model using \textbf{PyTorch}, including a Variational Autoencoder (VAE), \textbf{Diffusion U-Net} with timestep embeddings and self-attention, and various scheduling techniques
    \item Trained the \textbf{VAE} on the Fashion MNIST dataset using reconstruction loss and \textbf{KL-Divergence loss}. Implemented both unconditional and conditional \textbf{Denoising Diffusion Probabilistic Models (DDPM)} on CIFAR-10 dataset
    \item Implemented \textbf{latent diffusion} by encoding images to latent representations using the \textbf{Hugging Face diffuser's }VAE Trained a DDPM on these latents using the LSUN churches and bedrooms dataset to generate high-quality images
    % \item Compared the quality of the generations to the original using metrics like \textbf{Fréchet Inception Distance (FID)} scores
\end{itemize}

\vspace{\baselineskip}
\vspace{-15pt}
\noindent\textbf{\large Discrete Event Simulator for Bitcoin Network}\hfill{\sl \small (Spring 2023)}\\
{\it Guide: Prof. Vinay J. Ribeiro} $|$ {\it Course Project : Introduction to Blockchains and Smart Contracts } \hfill{\it IIT Bombay}\\
\vspace{-17pt}
\begin{itemize}[itemsep = -1.4 mm, leftmargin=*]
    \item Implemented a discrete event simulator for the Bitcoin Network and \textbf{analyzed the forking} and length of the main chain. Additionally, simulated \textbf{selfish mining} and \textbf{stubborn mining} attacks on the network by an adversary node 
    \item Analyzed the \textbf{adversary's relative profitability} under various factors like hashing power and network latency etc.
    \item Utilized the \textbf{Networkx library} to create a connected \textbf{P2P network} and generated visual representations of the blockchain. Used the \textbf{SimPy library} to maintain a \textbf{global clock} and simulate the \textbf{mining and transaction events}
\end{itemize}

\vspace{\baselineskip}
\vspace{-15pt}
\noindent\textbf{\large KYC-Website}\hfill{\sl \small (Summer 2023)}\\
{\it Self Project}
\\\vspace{-17pt}
\begin{itemize}[itemsep = -1.4 mm, leftmargin=*]
    \item Developed a secure web application using \textbf{Node.js, Express.js}, and \textbf{MongoDB} for KYC verification. Integrated \textbf{easy-ocr} library for \textbf{ID information extraction} and \textbf{face-recognition library} for \textbf{real-time face matching}
    \item Implemented \textbf{full-stack development} with \textbf{Bootstrap, EJS, and Passport.js} for a web application with secure authentication. Utilized \textbf{FastAPI} to wrap ML components, ensuring seamless communication with ML API servers.
\end{itemize}

\vspace{\baselineskip}
\vspace{-15pt}
\noindent\textbf{\large Deep Learning}\hfill{\sl \small (Summer 2023)}\\
{\it Self Project}
\\\vspace{-17pt}
\begin{itemize}[itemsep = -1.4 mm, leftmargin=*]
    \item Implemented and trained \textbf{Google's Deeppose}, a deep learning model for \textbf{human pose estimation} on LSP dataset
    \item Implemented a \textbf{Cycle-GAN} architecture for image-to-image translation, enabling conversion between two classes 
    \item Trained an agent to play \textbf{lunar lander} game using \textbf{Deep Q-Network (DQN)}, a reinforcement learning algorithm
    \item Implemented \textbf{neural style art transfer} using \textbf{VGG19} to combine the content of one image with the style of another
    \item Implemented the \textbf{U-Net} architecture and applied it to CARLA, a self-driving car dataset for \textbf{semantic segmentation}
    \item Implemented \textbf{ResNets} from scratch and utilized \textbf{transfer learning} for image classification and recognition tasks
\end{itemize}

\vspace{-20pt}
\noindent\section*{\color{mycolor}\LARGE Technical Skills\xfilll[0pt]{0.5pt}}
\vspace{-14pt}
\noindent\begin{tabular}{p{4.5cm} p{12.8cm}}
    \textbf{Programming}                        & C, C++, Python, Bash, Solidity, Java, JavaScript, VHDL, Sed, Awk                                    \\
    \vspace{-7pt} \textbf{Data Science} & \vspace{-7pt} Tensorflow, Pytorch, Keras, Trax, Scikit-learn, OpenCV, NumPy, Pandas, Matplotlib  \\
    \vspace{-7pt} \textbf{Software \& Tools} & \vspace{-7pt} MATLAB, Git, \LaTeX{}, Docker, Wireshark, Z3, Doxygen, Sphinx, Ngingx, FastAPI  \\
     \vspace{-7pt} \textbf{Web Development}   &  \vspace{-7pt} HTML5, CSS, JavaScript, BootStrap, jQuery, Node.js, Express.js, SQL, MongoDB                                      \\
    % \hline
\end{tabular}

\vspace{-17pt}
\noindent\section*{\color{mycolor}\LARGE Extracurricular\xfilll[0pt]{0.5pt}}
\vspace{-12pt}
\begin{itemize}[itemsep = -1.4 mm, leftmargin=*]
    \item Mentored two groups of students during the \textbf{SoC (Summer of Code)} program conducted by \textbf{WNCC, IITB} \hfill{\sl \small (2023)}
    \item Successfully completed one year under \textbf{National Sports Organization(NSO)} in \textbf{Chess} at IIT Bombay\hfill{\sl \small (2022)}
    \item Pitched a \textbf{Business Model Canvas} for a startup in the health sector which entailed making online\\ ambulance
          bookings, for the EnB Buzz contest conducted by the \textbf{Entrepreneurship cell of IIT Bombay}\hfill{\sl \small (2021)}
    \item Participated in a team of 3 and wrote a working script and successful submission in \textbf{Google Hashcode}\hfill{\sl \small (2021)}
    \item Worked as team of 4 to make a remote controlled bot using ESP32 for XLR8 - an event of \textbf{ERC, IITB}\hfill{\sl \small (2022)}

\end{itemize}
\end{document}

% \vspace{\baselineskip}
% \vspace{-15pt}
% \noindent\textbf{\large FastChat} \hfill{\sl \small Autumn 2022}\\
% {\it Guide: Prof. Kavi Arya} $|$ {\it Ongoing Course Project : Software Systems Lab } \hfill{\it IIT Bombay}\\
% \vspace{-17pt}
% \begin{itemize}[itemsep = -1.4 mm, leftmargin=*]
%     \item Developing a messaging software with \textbf{end-to-end encryption} by using \textbf{RSA+AES} to encode the messages and both \textbf{group chat} and \textbf{individual chat} support using \textbf{python socket library} and \textbf{PostgreSQL} database
%     \item Implementing a \textbf{load balancer} with \textbf{least connect strategy} using \textbf{bash} to distribute load among multiple servers
% \end{itemize}



% \vspace{\baselineskip}
% \vspace{-15pt}
% \noindent\textbf{\large Rail Planner}\hfill{\sl \small Autumn 2022}\\
% {\it Guide: Prof. Supratik Chakraborty} $|$ {\it Course Project : Data Structures and Algorithms Lab} \hfill{\it IIT Bombay}
% \\\vspace{-17pt}
% \begin{itemize}[itemsep = -1.4 mm, leftmargin=*]
%     \item Developing a railway planner using algorithms such as \textbf{Merge Sort, KMP, Quicksort}, etc.
%     \item Utilising Data Structures such as \textbf{linked lists, Binary Search Trees, AVL Trees, Hash tables, Tries}, etc.
% \end{itemize}


% \vspace{\baselineskip}
% \vspace{-15pt}
% \noindent\textbf{\large Machine Learning}\hfill{\sl \small Autumn 2022}\\
% {\it Self Project}
% \\\vspace{-17pt}
% \begin{itemize}[itemsep = -1.4 mm, leftmargin=*]
%     \item Learnt about the various machine learning algorithms such as \textbf{linear and logistic regressions, clustering using K-means, K-nearest neighbors and decision trees} and implemented them from scratch using numpy and pandas
%     \item Proficiency in using \textbf{cross-validation and hyperparameter tuning} to optimize machine learning models
%     \item Used \textbf{scikit-learn and XGBClassifer and XGBRegressor} to implement various types of classifiers and regressors to predict and classify various types of data such as \textbf{classifying flower species and predicting house prices}
% \end{itemize}

% \vspace{\baselineskip}
% \vspace{-15pt}
% \noindent\textbf{\large Lunar Lander using Deep Reinforcement Learning}\hfill{\sl \small Autumn 2022}\\
% {\it Self Project}
% \\\vspace{-17pt}
% \begin{itemize}[itemsep = -1.4 mm, leftmargin=*]
%     \item Implemented a \textbf{Deep Q learning} algorithm with \textbf{experience replay} using \textbf{TensorFlow} to train a lunar lander
%     \item Used \textbf{Tensorflow Core} to define a \textbf{custom loss function} and a \textbf{custom training loop} using \textbf{GradientTape} to train the model. Used \textbf{epsilon greedy policy} to select the action with some amount of random decisions
%     \item \textbf{Tuned and optimized model hyperparameters}, including learning rate, batch size, and number of episodes, epilon, gamma, number of timesteps to achieve the best results and solved the environment within 500 episodes
% \end{itemize}


% \vspace{\baselineskip}
% \vspace{-15pt}
% \noindent\textbf{\large Deep Learning and Neural Networks}\hfill{\sl \small Autumn 2022}\\
% {\it Self Project}
% \\\vspace{-17pt}
% \begin{itemize}[itemsep = -1.4 mm, leftmargin=*]
%     \item Made a \textbf{CNN} to classify images of handwritten digits using MNIST dataset and \textbf{TensorFlow, Keras and PyTorch} and compared their performance. Also made a \textbf{GUI} to draw digits using \textbf{python Tkinter} and classify them
%     \item Made many different types of \textbf{CNNs} to classify various types of data such as \textbf{Traffic signs recogninition, Crack detection, Smile detection, Hand sign recogninition} using \textbf{PyTorch} and \textbf{Keras Sequential API}
%     \item Successfully implemented \textbf{transfer learning} to train a \textbf{pretrained MobileNetV2} to classify images of \textbf{alpacas}
%     \item Implemented \textbf{ResNet50's architecture} from scratch using \textbf{Keras API} and trained it on a hand sign dataset
% \end{itemize}



% \vspace{-20pt}
% \noindent\section*{\color{mycolor}\LARGE Technical Skills\xfilll[0pt]{0.5pt}}
% \vspace{-12pt}
% \noindent\begin{tabular}{p{4.5cm} p{12.8cm}}
%     \textbf{Programming Languages:}                        & C++, Python, MATLAB, Java, Bash, Solidity, Sed, AWK   \vspace{1pt}                                          \\
%     \hline
%     \vspace{1pt} \vspace{-8pt} \textbf{Software \& Tools:} & \vspace{1pt} \vspace{-8pt} Tensorflow, Pytorch, Keras, Scikit-learn, OpenCV, Seaborn, Git, \LaTeX{}, MySQL, \\ & NumPy, Pandas, Matplotlib, Doxygen, Sphinx \vspace{1pt} \\
%     \hline
%     \vspace{1pt} \vspace{-8pt} \textbf{Web Development:}   & \vspace{1pt} \vspace{-8pt} HTML, CSS, JavaScript, BootStrap                                                 \\
    %\textbf{Software} & Git, \LaTeX, AutoCAD
    %\textbf{Web Development} & Django, HTML, CSS, JavaScript, PHP, Bootstrap, Jquery, Android Studio  \\
    %   \textbf{Extra Courses} & Probability theory, Advance Network Security and Cryptography, Derivative Pricing
% \end{tabular}

% \vspace{\baselineskip}
% \vspace{-15pt}
% \noindent\textbf{\large Generating Representative Images from a Sample} \hfill{\sl \small Autumn 2022}\\
% {\it Guide: Prof. Suyash Awate} $|$ {\it Ongoing Course Project : Data Analysis and Interpretation } \hfill{\it IIT Bombay}
% \vspace{-3pt}
% \begin{itemize}[itemsep = -1.4 mm, leftmargin=*]
%     \item Used \textbf{MATLAB} to generate new
%           representative fruit images using \textbf{Principal Component Analysis (PCA)}
%     \item Used PCA to optimally reduce the
%           dimensionality of digits from the \textbf{MNIST Database} and reconstruct the image
% \end{itemize}
% \vspace{\baselineskip}
% \vspace{-15pt}
% \noindent\textbf{\large Multiplayer Tic-Tac-Toe} \hfill{\sl \small Autumn 2022}\\
% {\it Guide: Prof. Kavi Arya} $|$ {\it Course Project : Software Systems Lab } \hfill{\it IIT Bombay}\\
% \vspace{-20pt}
% \begin{itemize}[itemsep = -1.4 mm, leftmargin=*]
%     \item Used \textbf{Java Socket Programming} for \textbf{inter process communication} using the \textbf{peer-to-peer model}
%     \item Created the tic tac toe game using this model and handled various newtork and \textbf{IOStream exceptions}
% \end{itemize}
% \vspace{\baselineskip}
% \vspace{-15pt}
% \noindent\textbf{\large Bubble Trouble} \hfill{\sl \small Autumn 2021}\\
% {\it Guide: Prof. Parag Chaudhuri} $|$ {\it Course Project : Computer Programming and Utilization } \hfill{\it IIT Bombay}
% \vspace{-7pt}
% \begin{itemize}[itemsep = -1.4 mm, leftmargin=*]
%     \item Developed a video game using the \textbf{simplecpp graphics library} and object oriented programming in \textbf{C++} with a
%           physics simulation to model the motion of bubbles along with features such as timers, health bars, levels and scores
% \end{itemize}