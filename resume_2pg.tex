\documentclass[a4paper,10pt]{article}
\usepackage[a4paper,bottom = 6.4 mm,left = 14.11 mm,right = 14.11 mm,top = 14.11 mm]{geometry}
\usepackage{graphicx}
\usepackage{amsmath}
\usepackage{array}
\usepackage{enumitem}
\usepackage{wrapfig}
\usepackage{microtype}
\usepackage{xcolor}
\usepackage{titlesec}
\usepackage{textcomp}
\usepackage[colorlinks=false]{hyperref}
\usepackage{verbatim}
\usepackage{titlesec}
\newcommand{\xfilll}[2][1ex]{
\dimen0=#2\advance\dimen0 by #1
\leaders\hrule height \dimen0 depth -#1\hfill}
\titleformat{\section}{\large\scshape\raggedright}{}{0em}{}
\renewcommand\labelitemi{\raisebox{0.4ex}{\tiny$\bullet$}}
\renewcommand{\labelitemii}{$\cdot$}
\pagenumbering{gobble}
\definecolor{mycolor}{RGB}{51, 102, 204}
\begin{document}
\vspace*{56.28 mm}
\noindent Pursuing a \textbf{Minor} degree in \textbf{Artificial Intelligence} and \textbf{Data Science} from \textbf{C-MInDS, IIT Bombay}
\vspace{-15pt}
\noindent\section*{{\color{mycolor}\LARGE Scholastic Achievements\xfilll[0pt]{0.5pt}}}
\vspace{-10pt}

\begin{itemize}[itemsep = -0.9 mm, leftmargin=*]
    \item Achieved \textbf{99.81 Percentile} in {\bf JEE-Main} out of over 1 million candidates\hfill{\sl \small (2021)}
    \item  Secured {\bf All India Rank 1207} in {\bf JEE-Advanced} out of over 0.14 million candidates\hfill{\sl \small (2021)}
    \item Secured {\bf AP grade} for excellent performance in \textbf{PH 108}, awarded to 27 out of over 1300 students \hfill{\sl \small (2022)}
    \item One of the \textbf{17 out of 1400+} students to secure a \textbf{Branch Change} to the department of \textbf{CSE}\hfill{\sl \small (2022)}
    \item Secured {\bf AIR 275} in the prestigious \textbf{KVPY SX} and awarded fellowship by \textbf{IISC Bangalore}\hfill{\sl \small (2021)}


\end{itemize}
\vspace{-20pt}
\noindent\section*{{\color{mycolor}\LARGE Work Experience\xfilll[0pt]{0.5pt}}}
\vspace{-10pt}
\noindent\textbf{\large Applied AI Researcher at Brance Technologies}\hfill{\sl \small (Summer 2023)}\\
\vspace{-15pt}
\begin{itemize}[itemsep = -0.65 mm, leftmargin=*]
    \item Developed and implemented performant chatbot systems using \textbf{vector embeddings} for data retrieval and \textbf{Large Language Models} for question-answering, resulting in significantly improved user experience and engagement
    \item Utilized \textbf{Haystack} framework and \textbf{FAISS} to efficiently index and store proprietary data, leveraging \textbf{vectorDBs} to store the embeddings. Utilized \textbf{Hugging Face models} to form the entire \textbf{retrieval pipeline}, including \textbf{document reranking}, resulting in improved relevance and accuracy of the retrieved information for chatbot responses
    \item Implemented \textbf{Locality-Sensitive Hashing (LSH)} to create a highly \textbf{performant caching system} to cache user queries that utilizes \textbf{semantic search}, optimizing the speed, accuracy, and efficiency of data retrieval for chatbot
    \item Leveraged \textbf{Nginx} and \textbf{FastAPI} on an \textbf{AWS EC2} instance to ensure seamless communication, and reduced latency for the chatbot system. Utilized \textbf{async calls} and FastAPI's scalability for smooth data retrieval and processing
\end{itemize}
\vspace{-20pt}
\noindent\section*{\color{mycolor}\LARGE Key Projects\xfilll[0pt]{0.5pt}}
\vspace{-10pt}
\noindent\textbf{\large Stable Diffusion from Scratch}\hfill{\sl \small (Summer 2023)}\\
{\it Self Project}
\\\vspace{-15pt}
\begin{itemize}[itemsep = -0.65 mm, leftmargin=*]
    \item Used \textbf{PyTorch} to independently develop and implement \textbf{each component} of a \textbf{Stable Diffusion} model on smaller datasets before using the \textbf{Hugging Face's diffuser} library to implement the final diffusion model on a larger dataset
    \item Implemented a \textbf{Variational Autoencoder (VAE)} and trained it on the \textbf{Fashion MNIST} dataset using \textbf{reconstruction loss} and \textbf{KL-Divergence loss}, enabling accurate reconstruction of inputs and latent space interpolation
    \item Implemented a \textbf{Diffusion U-Net} architecture with \textbf{timestep embeddings} and \textbf{self-attention} and used it to implement both unconditional and conditional \textbf{DDPM(Denoising Diffusion Probabilistic Models)} on the \textbf{CIFAR-10} dataset. Also used \textbf{various sampling techniques} to improve the quality of the generated images
    \item Implemented \textbf{latent diffusion} by utilizing the \textbf{diffuser's VAE} to encode images to \textbf{latent representations}, and subsequently trained a \textbf{DDPM} on these latents using the \textbf{LSUN churches and bedrooms datasets}  to generate high-quality images. Used \textbf{FID}(Fréchet Inception Distance) scores to evaluate the quality of the generated images
\end{itemize}
\vspace{\baselineskip}
\vspace{-15pt}
\noindent\textbf{\large Discrete Event Simulator for Bitcoin Network}\hfill{\sl \small (Spring 2023)}\\
{\it Guide: Prof. Vinay J. Ribeiro} $|$ {\it Course Project : Introduction to Blockchains and Smart Contracts } \hfill{\it IIT Bombay}\\
\vspace{-15pt}
\begin{itemize}[itemsep = -0.65 mm, leftmargin=*]
    \item Implemented a DES for the Bitcoin Network and \textbf{analyzed the forking} and length of the main chain. Additionally, simulated \textbf{selfish mining and stubborn mining attacks} on the network by an adversary node and analyzed the \textbf{adversary's relative profitability} under various factors such as hashing power, number of nodes and latency etc.
    \item Utilized the \textbf{Networkx library} to create a connected \textbf{P2P network} and generated visual representations of the blockchain. Used the \textbf{SimPy library} to maintain a \textbf{global clock} and simulate the \textbf{mining and transaction events}
\end{itemize}
\vspace{\baselineskip}
\vspace{-15pt}
\noindent\textbf{\large Layer 2 DAPP for Lightning Network Simulation}\hfill{\sl \small (Spring 2023)}\\
{\it Guide: Prof. Vinay J. Ribeiro} $|$ {\it Course Project : Introduction to Blockchains and Smart Contracts } \hfill{\it IIT Bombay}\\
\vspace{-15pt}
\begin{itemize}[itemsep = -0.65 mm, leftmargin=*]
    \item Developed a Layer 2 DAPP on the Ethereum blockchain, utilizing \textbf{Ganache/Truffle} to set up a \textbf{local Ethereum node}. Implemented the smart contract in \textbf{Solidity}, enabling the execution of transactions within the Layer 2 DAPP and facilitate the simulation of \textbf{Lightning Networks}, a Layer 2 scaling solution for the Ethereum blockchain
    \item Ran the simulation using an external Python script to perform various transaction scenarios and analyze their outcomes
\end{itemize}
\vspace{\baselineskip}
\vspace{-15pt}
\noindent\textbf{\large FastChat} \hfill{\sl \small (Autumn 2022)}\\
{\it Guide: Prof. Kavi Arya} $|$ {\it Course Project : Software Systems Lab } \hfill{\it IIT Bombay}\\
\vspace{-15pt}
\begin{itemize}[itemsep = -0.65 mm, leftmargin=*]
    \item Developing a messaging software with \textbf{end-to-end encryption} by using \textbf{RSA+AES} to encode the messages and both \textbf{group chat} and \textbf{individual chat} support using \textbf{python socket library} and \textbf{PostgreSQL} database
    \item Implementing a \textbf{load balancer} with \textbf{least connect strategy} using \textbf{bash} to distribute load among multiple servers, and focusing on obtaining \textbf{high throughput} while using only \textbf{limited resources} dedicated for the servers
    \item Used \textbf{bash} scripts to simulate common messaging patterns and calculate \textbf{throughput} and \textbf{latency} of the system
\end{itemize}
\vspace{\baselineskip}
\vspace{-15pt}
\noindent\textbf{\large KYC-Website}\hfill{\sl \small (Summer 2023)}\\
{\it Self Project}
\\\vspace{-15pt}
\begin{itemize}[itemsep = -0.65 mm, leftmargin=*]
    \item Developed and implemented KYC-Website, a secure web application utilizing \textbf{Node.js, Express.js, and MongoDB Atlas} for \textbf{KYC verification}, mimicking the KYC requirements for banking and financial institutions
    \item Utilized the \textbf{easy-ocr library} for \textbf{ID information extraction} and the \textbf{face-recognition library} for \textbf{real-time face matching} to automate the KYC verification process, ensuring swift \textbf{verification without human intervention}
    \item Learned about \textbf{full-stack development} by utilizing technologies such as \textbf{Bootstrap, EJS, Passport.js} to create a user-friendly and responsive web application with \textbf{secure authentication, form validation and user sessions}
    \item Utilized \textbf{FastAPI} to wrap and integrate the \textbf{machine learning components} of the project, ensuring seamless communication and efficient handling of requests between the \textbf{web application and the ML API servers}
\end{itemize}
\vspace{\baselineskip}
\vspace{-15pt}
\noindent\textbf{\large Deep Learning}\hfill{\sl \small (Autumn 2022)}\\
{\it Self Project}
\\\vspace{-15pt}
\begin{itemize}[itemsep = -0.65 mm, leftmargin=*]
    \item Made a \textbf{convolutional neural network} to classify images of handwritten digits using \textbf{MNIST} dataset and also made a \textbf{GUI} to draw digits using \textbf{python Tkinter} and classify them using the model. Made three different models using \textbf{TensorFlow Core, Keras Functional API and PyTorch} and compared their performance
    \item Made many different types of \textbf{CNNs} to classify various types of data such as \textbf{Traffic signs recogninition, Crack detection, Smile detection, Hand sign recogninition} using \textbf{PyTorch} and \textbf{Keras Sequential API}
    \item Successfully implemented \textbf{transfer learning} to train a \textbf{pretrained MobileNetV2} to classify images of \textbf{alpacas} and used \textbf{data augmentation} like random rotation and flipping resulting in a highly accurate model
    \item Implemented \textbf{ResNet50's architecture} from scratch using \textbf{Keras Functional API} and trained it on a hand sign dataset to classify images of 6 different classes. Compared its performance to a \textbf{pretrained ResNet50 model}
\end{itemize}
\vspace{-20pt}
\noindent\section*{\color{mycolor}\LARGE Other Projects\xfilll[0pt]{0.5pt}}
% \pagebreak
\vspace{\baselineskip}
\vspace{-15pt}
\noindent\textbf{\large Rail Planner}\hfill{\sl \small (Autumn 2022)}\\
{\it Guide: Prof. Supratik Chakraborty} $|$ {\it Course Project : Data Structures and Algorithms Lab} \hfill{\it IIT Bombay}
\\\vspace{-15pt}
\begin{itemize}[itemsep = -0.65 mm, leftmargin=*]
    \item Designed a simplified vesrion of a railway planner using various data structures and analyzed the space \& time complexity and the efficiency to demonstrate the \textbf{properties of different data structures in C++}
    \item Stored trains as a dictionary using \textbf{Hash Tables} and devised algorithms for fastest possible journies
    \item Used \textbf{BSTs and then AVL trees} for quick searching using the journey codes and used \textbf{Tries} to implement the autocompletion feature while searching for station names and added a feature to accept reviews for journies
    \item Used \textbf{Quicksort} to order trains by day and time, implemented the \textbf{KMP-string matching algorithm} for allowing review searches by using keywords and implemented \textbf{Heaps} to allow filtering the reviews by their rating
\end{itemize}
\vspace{\baselineskip}
\vspace{-15pt}
\noindent\textbf{\large Z3 SAT Solver} \hfill{\sl \small (Spring 2023)}\\
{\it Guide: Prof. Ashutosh Gupta} $|$ {\it Course Project : Logic for Computer Science } \hfill{\it IIT Bombay}
\vspace{-2pt}
\begin{itemize}[itemsep = -0.65 mm, leftmargin=*]
    \item Implemented and utilized \textbf{SAT solving techniques}, specifically leveraging the \textbf{Z3 theorem prover} in Python, to formulate an effective strategy for the game \textbf{Sliding-Solver} and also handle the case of \textbf{unsatisfiability} for the game
\end{itemize}
\vspace{\baselineskip}
\vspace{-15pt}
\noindent\textbf{\large Champsim Branch Predictor} \hfill{\sl \small (Spring 2023)}\\
{\it Guide: Prof. Ashutosh Gupta} $|$ {\it Course Project : Logic for Computer Science } \hfill{\it IIT Bombay}
\vspace{-2pt}
\begin{itemize}[itemsep = -0.65 mm, leftmargin=*]
    \item Implemented and utilized \textbf{SAT solving techniques}, specifically leveraging the \textbf{Z3 theorem prover} in Python, to formulate an effective strategy for the game \textbf{Sliding-Solver} and also handle the case of \textbf{unsatisfiability} for the game
\end{itemize}
\vspace{\baselineskip}
\vspace{-15pt}
\noindent\textbf{\large Swarcomm} \hfill{\sl \small (Spring 2023)}\\
{\it Guide: Prof. Ashutosh Gupta} $|$ {\it Course Project : Logic for Computer Science } \hfill{\it IIT Bombay}
\vspace{-2pt}
\begin{itemize}[itemsep = -0.65 mm, leftmargin=*]
    \item Implemented and utilized \textbf{SAT solving techniques}, specifically leveraging the \textbf{Z3 theorem prover} in Python, to formulate an effective strategy for the game \textbf{Sliding-Solver} and also handle the case of \textbf{unsatisfiability} for the game
\end{itemize}
% \vspace{\baselineskip}
% \vspace{-10pt}
% \noindent\textbf{\large Bubble Trouble} \hfill{\sl \small (Spring 2021)}\\
% {\it Guide: Prof. Parag Chaudhuri} $|$ {\it Course Project : Computer Programming and Utilization } \hfill{\it IIT Bombay}
% \vspace{-3pt}
% \begin{itemize}[itemsep = -0.65 mm, leftmargin=*]
%     \item Designed an interactive single player retro style game which impelements a bubble shooter to shoot random floating bubbles on the screen to demonstrate the \textbf{Object Oriented Paradigm in C++}
%     \item Implemented event-handling using \textbf{XEvent} object extensively used the \textbf{C++ STL} and the Simplecpp library that was developed in-house by the institute to add the various features of the game
%     \item Handled various events, assigning multiple responses by the game and designed the game for many levels of difficulty
% \end{itemize}

\vspace{-15pt}
\noindent\section*{\color{mycolor}\LARGE Technical Skills\xfilll[0pt]{0.5pt}}
\vspace{-9pt}
\noindent\begin{tabular}{p{4.5cm} p{12.8cm}}
    \textbf{Programming}                        & C/C++, Python, Bash, Solidity, Java, JavaScript, VHDL, Sed, Awk     \vspace{2pt}                                     \\
    \hline
    \vspace{2pt} \vspace{-7pt} \textbf{Data Science} & \vspace{2pt} \vspace{-7pt} Tensorflow, Pytorch, Keras, Trax, Scikit-learn, OpenCV, NumPy, Pandas, Matplotlib  \\
    \hline
    \vspace{2pt} \vspace{-7pt} \textbf{Software \& Tools} & \vspace{2pt} \vspace{-7pt} MATLAB, Git, \LaTeX{}, Docker, Wireshark, Z3, Doxygen, Sphinx, Ngingx, FastAPI  \vspace{2pt}\\
    \hline
    \vspace{2pt} \vspace{-7pt} \textbf{Web Development}   & \vspace{2pt} \vspace{-7pt} HTML5, CSS, JavaScript, BootStrap, jQuery NodeJS, ExpressJS, SQL, MongoDB                                      \\
    % \hline
\end{tabular}
\vspace{-12pt}

%\vspace{-17pt}

\noindent\section*{\color{mycolor}\LARGE Key Courses Undertaken\xfilll[0pt]{0.5pt}}
\vspace{-8pt}
\noindent\begin{tabular}{m{36mm} m{13.7cm}}
    \textbf{Mathematics}  \vspace{2pt}                   & Calculus, Linear Algebra, Discrete Structures, Differential Equations, Optimization Models, Logic for Computer Science, Game Theory and Decision Analysis      \vspace{2pt}                                                                                \\
    \hline
    \vspace{3pt} \vspace{-1pt} \textbf{Computer Science} & \vspace{3pt} \vspace{-1pt}  Data Structures and Algorithms\textsuperscript{\#}, Data Analysis and Interpretation, Software Systems Lab,
    Computer Networks\textsuperscript{\#}, Digital Logic Design\textsuperscript{\#}, Design and Analysis of Algorithms,
     Introduction to Blockchains Cryptocurrencies and Smart Contracts, Computer Vision\vspace{2pt}                                                                                               \\
    % \hline
    % \vspace{3pt} \vspace{-1pt} \textbf{Miscellaneous}    & \vspace{3pt} \vspace{-1pt} Game Theory and Decision Analysis                                                                                                                                                                          \\                                                                                                                    \\
    % \textbf{Online Courses}   & Machine Learning
\end{tabular}
\vspace{-5pt}
\begin{flushright}
    % \sl\small(* to be completed by April 2023)\\ 
    \sl \small (\# Theory + Lab) \\
\end{flushright}
\vspace{-15pt}
\noindent\section*{\color{mycolor}\LARGE Extracurricular\xfilll[0pt]{0.5pt}}
\vspace{-7pt}
\begin{itemize}[itemsep = -0.65 mm, leftmargin=*]
    \item Mentored two groups of students during the \textbf{SoC (Summer of Code)} program conducted by \textbf{WNCC, IITB} guiding them through deep learning projects and assisting in the implementation of cutting-edge research papers.  \hfill{\sl \small (2023)}
    \item Successfully completed one year under \textbf{National Sports Organization(NSO)} in \textbf{Chess} at IIT Bombay\hfill{\sl \small (2022)}
    \item Pitched a \textbf{Business Model Canvas} for a startup in the health sector which entailed making online ambulance
          bookings, for the EnB Buzz competition conducted by the \textbf{Entrepreneurship cell of IIT Bombay}\hfill{\sl \small (2021)}
    \item Worked in a team of 4 to make an ESP32 \textbf{WiFi-controlled} bot for \textbf{XLR8} conducted by \textbf{ERC, IITB}\hfill{\sl \small (2022)}


\end{itemize}
\end{document}

% \vspace{\baselineskip}
% \vspace{-20pt}
% \noindent\textbf{\large Machine Learning}\hfill{\sl \small (Autumn 2022)}\\
% {\it Self Project}
% \\\vspace{-15pt}
% \begin{itemize}[itemsep = -0.65 mm, leftmargin=*]
%     \item Experience in machine learning frameworks such as \textbf{scikit-learn, XGBoost, PyTorch, TensorFlow, and Keras}.
%     \item Using Python packages such as \textbf{numpy, pandas, matplotlib and seaborn} for data manipulation and analysis. Learnt about \textbf{feature engineering} and feature selection techniques to improve the performance of the models
%     \item Learnt about the various machine learning algorithms such as \textbf{linear and logistic regressions, clustering using K-means, K-nearest neighbors and decision trees} and implemented them from scratch using numpy and pandas
%     \item Proficiency in using \textbf{cross-validation and hyperparameter tuning} to optimize machine learning models
%     \item Used \textbf{scikit-learn and XGBClassifer and XGBRegressor} to implement various types of classifiers and regressors to predict and classify various types of data such as \textbf{classifying flower species and predicting house prices}
% \end{itemize}

% \vspace{\baselineskip}
% \vspace{-15pt}
% \noindent\textbf{\large Human Pose Estimation}\hfill{\sl \small (Summer 2023)}\\
% {\it Self Project}
% \\\vspace{-15pt}
% \begin{itemize}[itemsep = -0.65 mm, leftmargin=*]
%     \item Used \textbf{Pygame} and \textbf{OpenAI's Gym} to train a lunar lander game \textbf{Deep Q-Learning with Experience Replay}. Used the \textbf{Sequential API} of the Keras library to define the \textbf{Q-network} and the \textbf{target Q-network}
%     \item Used \textbf{Tensorflow Core} to define a \textbf{custom loss function} and a \textbf{custom training loop} using \textbf{GradientTape} to train the model. Used \textbf{epsilon greedy policy} to select the action with some amount of random decisions
%     \item Utilized a \textbf{deque} for storing the experience buffer and used \textbf{experience replay} and \textbf{soft update} of the Q targets to stabilize the learning process and improve the model's convergence towards an optimal solution
%     \item \textbf{Tuned and optimized model hyperparameters}, including learning rate, batch size, and number of episodes, epilon, gamma, number of timesteps to achieve the best results and solved the environment within 500 episodes
% \end{itemize}

% \vspace{\baselineskip}
% \vspace{-15pt}
% \noindent\textbf{\large Cycle-GANs}\hfill{\sl \small (Summer 2023)}\\
% {\it Self Project}
% \\\vspace{-15pt}
% \begin{itemize}[itemsep = -0.65 mm, leftmargin=*]
%     \item Used \textbf{Pygame} and \textbf{OpenAI's Gym} to train a lunar lander game \textbf{Deep Q-Learning with Experience Replay}. Used the \textbf{Sequential API} of the Keras library to define the \textbf{Q-network} and the \textbf{target Q-network}
%     \item Used \textbf{Tensorflow Core} to define a \textbf{custom loss function} and a \textbf{custom training loop} using \textbf{GradientTape} to train the model. Used \textbf{epsilon greedy policy} to select the action with some amount of random decisions
%     \item Utilized a \textbf{deque} for storing the experience buffer and used \textbf{experience replay} and \textbf{soft update} of the Q targets to stabilize the learning process and improve the model's convergence towards an optimal solution
%     \item \textbf{Tuned and optimized model hyperparameters}, including learning rate, batch size, and number of episodes, epilon, gamma, number of timesteps to achieve the best results and solved the environment within 500 episodes
% \end{itemize}

% \vspace{\baselineskip}
% \vspace{-15pt}
% \noindent\textbf{\large Deep Learning}\hfill{\sl \small (Autumn 2022)}\\
% {\it Self Project}
% \\\vspace{-15pt}
% \begin{itemize}[itemsep = -0.65 mm, leftmargin=*]
%     \item Made a \textbf{convolutional neural network} to classify images of handwritten digits using \textbf{MNIST} dataset and also made a \textbf{GUI} to draw digits using \textbf{python Tkinter} and classify them using the model. Made three different models using \textbf{TensorFlow Core, Keras Functional API and PyTorch} and compared their performance
%     \item Made many different types of \textbf{CNNs} to classify various types of data such as \textbf{Traffic signs recogninition, Crack detection, Smile detection, Hand sign recogninition} using \textbf{PyTorch} and \textbf{Keras Sequential API}
%     \item Successfully implemented \textbf{transfer learning} to train a \textbf{pretrained MobileNetV2} to classify images of \textbf{alpacas} and used \textbf{data augmentation} like random rotation and flipping resulting in a highly accurate model
%     \item Implemented \textbf{ResNet50's architecture} from scratch using \textbf{Keras Functional API} and trained it on a hand sign dataset to classify images of 6 different classes. Compared its performance to a \textbf{pretrained ResNet50 model}
% \end{itemize}

% \vspace{\baselineskip}
% \vspace{-15pt}
% \noindent\textbf{\large Bubble Trouble} \hfill{\sl \small Autumn 2021}\\
% {\it Guide: Prof. Parag Chaudhuri} $|$ {\it Course Project : Computer Programming and Utilization } \hfill{\it IIT Bombay}
% \vspace{-2pt}
% \begin{itemize}[itemsep = -0.65 mm, leftmargin=*]
%     \item Developed a video game using the \textbf{simplecpp graphics library} and object oriented programming in \textbf{C++} with a
%           physics simulation to model the motion of bubbles along with features such as timers, health bars, levels and scores
% \end{itemize}

% \vspace{\baselineskip}
% \vspace{-15pt}
% \noindent\textbf{\large Lunar Lander using Deep Reinforcement Learning}\hfill{\sl \small (Autumn 2022)}\\
% {\it Self Project}
% \\\vspace{-15pt}
% \begin{itemize}[itemsep = -0.65 mm, leftmargin=*]
%     \item Used \textbf{Pygame} and \textbf{OpenAI's Gym} to train a lunar lander game \textbf{Deep Q-Learning with Experience Replay}. Used the \textbf{Sequential API} of the Keras library to define the \textbf{Q-network} and the \textbf{target Q-network}
%     \item Used \textbf{Tensorflow Core} to define a \textbf{custom loss function} and a \textbf{custom training loop} using \textbf{GradientTape} to train the model. Used \textbf{epsilon greedy policy} to select the action with some amount of random decisions
%     \item Utilized a \textbf{deque} for storing the experience buffer and used \textbf{experience replay} and \textbf{soft update} of the Q targets to stabilize the learning process and improve the model's convergence towards an optimal solution
%     \item \textbf{Tuned and optimized model hyperparameters}, including learning rate, batch size, and number of episodes, epilon, gamma, number of timesteps to achieve the best results and solved the environment within 500 episodes
% \end{itemize}

% \vspace{\baselineskip}
% \vspace{-15pt}
% \noindent\textbf{\large Monte Carlo Analysis of Statistical Theorems} \hfill{\sl \small (Autumn 2022)}\\
% {\it Guide: Prof. Suyash Awate} $|$ {\it Course Project : Data Analysis and Interpretation } \hfill{\it IIT Bombay}
% \vspace{-3pt}
% \begin{itemize}[itemsep = -0.65 mm, leftmargin=*]
%     \item Used \textbf{MATLAB} to implement a Monte Carlo simulation of a given Probability distribution
%     \item Plotted the probability and cumulative distribution functions of various distributions and empirically verified various statistical theorems such as the law of large numbers, Poison thinning and the
%           Gaussian nature of the Random Walk
% \end{itemize}
% \vspace{\baselineskip}
% \vspace{-12pt}
% \noindent\textbf{\large Text File Editors}\hfill{\sl \small (Autumn 2022)}\\
% {\it Guide: Prof. Kavi Arya} $|$ {\it Course Project : Software Systems Lab } \hfill{\it IIT Bombay}
% \\\vspace{-15pt}
% \begin{itemize}[itemsep = -0.65 mm, leftmargin=*]
%     \item Developed an analog to the Linux Command Line utility \textbf{wc command} using the \textbf{awk programming language}
%           that counts the number of characters, words and lines in a text file and also accepts flags similar to wc command
%     \item Developed a program to check for valid email addresses using \textbf{sed} with pattern matching using \textbf{regular expressions}
%     \item Implemented a \textbf{csv file editor} that formats columns based on customisable properties such as date, time and name
%     \item Developed a program which changes the base of the number to a different given base using \textbf{bash scripting and awk}
%     \item Developed a program to \textbf{encrypt} a piece of text when the words to encrypt and their corresponding cipher is given
% \end{itemize}

% \vspace{\baselineskip}
% \vspace{-15pt}
% \noindent\textbf{\large Multiplayer Tic-Tac-Toe} \hfill{\sl \small (Autumn 2022)}\\
% {\it Guide: Prof. Kavi Arya} $|$ {\it Course Project : Software Systems Lab } \hfill{\it IIT Bombay}\\
% \vspace{-15pt}
% \begin{itemize}[itemsep = -0.65 mm, leftmargin=*]
%     \item Used \textbf{Java Socket Programming} for \textbf{inter process communication} using the \textbf{peer-to-peer model}
%     \item Created the tic tac toe game using this model and handled various newtork and \textbf{IOStream exceptions}
% \end{itemize}

% \vspace{\baselineskip}
% \vspace{-15pt}
% \noindent\textbf{\large Personal Website}\hfill{\sl \small (Autumn 2022)}\\
% {\it Guide: Prof. Kavi Arya} $|$ {\it Course Project : Software Systems Lab } \hfill{\it IIT Bombay}
% \\\vspace{-15pt}
% \begin{itemize}[itemsep = -0.65 mm, leftmargin=*]
%     \item Made a personal website to be hosted on the CSE department server using \textbf{HTML and CSS and JavaScript}
%     \item Added various advanced \textbf{CSS} features animations, transitions, static scroll images, modals, checkboxes and slideshows
%     \item Used \textbf{JavaScript} to make the website interactive, gauge user-choices and render web-pages accordingly and deployed the website on an SSH server; used \textbf{BootStrap} to impelement standard navigation bars, footers and other features
% \end{itemize}

% \vspace{\baselineskip}
% \vspace{-15pt}
% \noindent\textbf{\large Generating Representative Images from a Sample} \hfill{\sl \small (Autumn 2022)}\\
% {\it Guide: Prof. Suyash Awate} $|$ {\it Ongoing Course Project : Data Analysis and Interpretation } \hfill{\it IIT Bombay}
% \vspace{-3pt}
% \begin{itemize}[itemsep = -0.65 mm, leftmargin=*]
%     \item Used \textbf{MATLAB} to implement a Monte Carlo simulation of a given distribution and plotted the PDF and CDF
%     \item Used \textbf{MATLAB} to use a data set of images of various fruits and sampled random images to generate new
%           representative fruit images using \textbf{Principal Component Analysis (PCA)}. Also used PCA to analyse images of handwritten digits from the \textbf{MNIST Database} and optimally \textbf{reduce the
%               dimensionality} and \textbf{reconstruct} the images
% \end{itemize}