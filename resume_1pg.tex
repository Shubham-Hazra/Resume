\documentclass{article}
\usepackage[a4paper,bottom = 6.4 mm,left = 14.11 mm,right = 14.11 mm,top = 14.11 mm]{geometry}
\usepackage{graphicx}
\usepackage{amsmath}
\usepackage{array}
\usepackage{enumitem}
\usepackage{wrapfig}
\usepackage{microtype}
\usepackage{xcolor}
\usepackage{titlesec}
\usepackage{textcomp}
\usepackage[colorlinks=false]{hyperref}
\usepackage{verbatim}
\usepackage{titlesec}
\newcommand{\xfilll}[2][1ex]{
\dimen0=#2\advance\dimen0 by #1
\leaders\hrule height \dimen0 depth -#1\hfill}
\titleformat{\section}{\large\scshape\raggedright}{}{0em}{}
\renewcommand\labelitemi{\raisebox{0.4ex}{\tiny$\bullet$}}
\renewcommand{\labelitemii}{$\cdot$}
\pagenumbering{gobble}
\begin{document}


\vspace*{56.28 mm}
\noindent Pursuing a \textbf{Minor} degree in \textbf{ Artificial Intelligence} and \textbf{Data Science} from \textbf{C-MInDS, IIT Bombay}
\vspace{-10pt}
\noindent\section*{{\color{black!20!blue}\LARGE Scholastic Achievements}\xfilll[0pt]{0.5pt}}
\vspace{-10pt}

\begin{itemize}[itemsep = -0.65 mm, leftmargin=*]
    \item Achieved \textbf{99.81 Percentile} in {\bf JEE-Main} out of over 1 million candidates\hfill{\sl \small (2021)}
    \item  Secured {\bf All India Rank 1207} in {\bf JEE-Advanced} out of over 0.14 million candidates\hfill{\sl \small (2021)}
    \item Secured {\bf AP(Advanced Performer) grade} for excellent performance in \textbf{PH 108-Basics of Electricity \& Magnetism}, awarded to 27 out of over 1300 students taking the course\hfill{\sl \small (2022)}
    \item One of the \textbf{17 out of 1400+} students to secure a \textbf{Change of Branch} to the department of \textbf{Computer Science and Engineering} owing to excellent academic performance in first year at IIT Bombay\hfill{\sl \small (2022)}
    \item Secured {\bf All India Rank 275} in the prestigious \textbf{KVPY (Kishore Vaigyanik Protsahan Yojna)} SX and awarded fellowship by the Department of Sciences, \textbf{Indian Institute of Science(IISC) Bangalore}\hfill{\sl \small (2021)}


\end{itemize}
\vspace{-18pt}
\vspace{0pt}
\noindent\section*{\color{black!20!blue}\LARGE Key Projects\xfilll[0pt]{0.5pt}}
\vspace{-7pt}
\noindent\textbf{\large FastChat} \hfill{\sl \small Autumn 2022}\\
{\it Guide: Prof. Kavi Arya} $|$ {\it Ongoing Course Project : Software Systems Lab } \hfill{\it IIT Bombay}\\
\vspace{-15pt}
\begin{itemize}[itemsep = -0.65 mm, leftmargin=*]
    \item Developing a messaging software by building a network of clients interacting via servers acting as mediators
    \item Focusing on obtaining \textbf{high throughput} while using only \textbf{limited resources} dedicated for the servers
    \item Ensuring \textbf{low latency} of individual message deliveries and \textbf{end-to-end encryption} between clients
    \item Using \textbf{python socket library} to develop the network, using \textbf{open source libraries} for authentication and communication, \textbf{PostgreSQL} database to store the data and \textbf{bash} for scripting and collecting results
    \item Adding flair to this web application by implementing an interactive frontend using \textbf{HTML, CSS and JavaScript}
\end{itemize}
\vspace{\baselineskip}
\vspace{-10pt}
\noindent\textbf{\large Rail Planner}\hfill{\sl \small Autumn 2022}\\
{\it Guide: Prof. Supratik Chakraborty} $|$ {\it Course Project : Data Structures and Algorithms Lab} \hfill{\it IIT Bombay}
\\\vspace{-15pt}
\begin{itemize}[itemsep = -0.65 mm, leftmargin=*]
    \item Designed a simplified vesrion of a railway planner using various data structures and analyzed the space, time complexity and the efficiency to demonstrate the \textbf{properties of different data structures in C++}
    \item Stored trains as a dictionary using \textbf{Hash Tables} and devised algorithms for fastest possible journies
    \item Used \textbf{BSTs and then AVL trees} for quick searching using the journey codes and used \textbf{Tries} to implement the autocompletion feature while searching for station names
    \item Used \textbf{Quicksort} to order trains by day and time, implemented the \textbf{KMP-string matching algorithm} for allowing review searches by using keywords and implemented \textbf{Heaps} to allow filtering the reviews by their rating
\end{itemize}
\vspace{\baselineskip}
\vspace{-10pt}
\noindent\textbf{\large Generating Representative Images from a Sample} \hfill{\sl \small Autumn 2022}\\
{\it Guide: Prof. Suyash Awate} $|$ {\it Ongoing Course Project : Data Analysis and Interpretation } \hfill{\it IIT Bombay}
\vspace{-3pt}
\begin{itemize}[itemsep = -0.65 mm, leftmargin=*]
    \item Used \textbf{MATLAB} to use a data set of images of various fruits and sampled random images to generate new
          representative fruit images using \textbf{Principal Component Analysis (PCA)}
    \item Used PCA to analyse images of handwritten digits from the \textbf{MNIST Database} and optimally reduce the
          dimensionality and reconstruct the image
    \item Implemented hyperplane fitting of 2 random variables and sampled points in the Euclidean Plane according to
          a given multivariate distribution
\end{itemize}
\vspace{\baselineskip}
\vspace{-10pt}
\noindent\textbf{\large Multiplayer Tic-Tac-Toe} \hfill{\sl \small Autumn 2022}\\
{\it Guide: Prof. Kavi Arya} $|$ {\it Course Project : Software Systems Lab } \hfill{\it IIT Bombay}\\
\vspace{-15pt}
\begin{itemize}[itemsep = -0.65 mm, leftmargin=*]
    \item Used \textbf{Java Socket Programming} for \textbf{inter process communication} using the \textbf{peer-to-peer model}
    \item Created the tic tac toe game using this model and handled various newtork and \textbf{IOStream exceptions}
\end{itemize}
\pagebreak
\vspace{\baselineskip}
\vspace{-10pt}
\noindent\textbf{\large Bubble Trouble} \hfill{\sl \small Autumn 2021}\\
{\it Guide: Prof. Parag Chaudhuri} $|$ {\it Course Project : Computer Programming and Utilization } \hfill{\it IIT Bombay}
\vspace{-3pt}
\begin{itemize}[itemsep = -0.65 mm, leftmargin=*]
    \item Designed an interactive single player retro style game which impelements a bubble shooter to shoot random floating bubbles on the screen to demonstrate the \textbf{Object Oriented Paradigm in C++}
    \item Implemented event-handling using \textbf{XEvent} object extensively used the \textbf{C++ STL} and the Simplecpp library that was developed in-house by the institute to add the various features of the game
    \item Handled various events, assigning multiple responses by the game and designed the game for many levels of difficulty
\end{itemize}

\vspace{-17pt}
\noindent\section*{\color{black!20!blue}\LARGE Technical Skills\xfilll[0pt]{0.5pt}}
\vspace{-9pt}
\noindent\begin{tabular}{p{4.5cm} p{13.5cm}}
    \textbf{Programming Languages:} & C++, Python, MATLAB, Java, Bash, Solidity, Sed, AWK                   \\
    \textbf{Software \& Tools:}     & Git, \LaTeX{}, MySQL, NumPy, Pandas, Matplotlib, Doxygen, Sphinx, gdb \\
    \textbf{Web Development:}       & HTML, CSS, JavaScript, BootStrap
    %\textbf{Software} & Git, \LaTeX, AutoCAD
    %\textbf{Web Development} & Django, HTML, CSS, JavaScript, PHP, Bootstrap, Jquery, Android Studio  \\
    %   \textbf{Extra Courses} & Probability theory, Advance Network Security and Cryptography, Derivative Pricing
\end{tabular}
\vspace{-12pt}
\noindent\section*{\color{black!20!blue}\LARGE Extracurricular\xfilll[0pt]{0.5pt}}
\vspace{-7pt}
\begin{itemize}[itemsep = -0.65 mm, leftmargin=*]
    \item Successfully completed one year under \textbf{National Sports Organization(NSO)} in \textbf{Chess} at IIT Bombay\hfill{\sl \small (2022)}
    \item Pitched a \textbf{Business Model Canvas} for a startup in the health sector which entailed making online ambulance
          bookings, for the EnB Buzz competition conducted by the \textbf{Entrepreneurship cell of IIT Bombay}\hfill{\sl \small (2021)}
    \item Participated in a team of 3 and wrote a working script and successful submission in \textbf{Google Hashcode 2021}\hfill{\sl \small (2021)}
    \item Worked as team of 4 to make a remote controlled bot using ESP32 for XLR8 - an event of \textbf{ERC, IITB}\hfill{\sl \small (2022)}

\end{itemize}
\end{document}
