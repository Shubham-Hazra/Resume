\documentclass[a4paper,10pt]{article}
\usepackage[a4paper,bottom = 6.4 mm,left = 14.11 mm,right = 14.11 mm,top = 14.11 mm]{geometry}
\usepackage{graphicx}
\usepackage{amsmath}
\usepackage{array}
\usepackage{enumitem}
\usepackage{wrapfig}
\usepackage{microtype}
\usepackage{xcolor}
\usepackage{titlesec}
\usepackage{textcomp}
\usepackage[colorlinks=false]{hyperref}
\usepackage{verbatim}
\usepackage{titlesec}
\newcommand{\xfilll}[2][1ex]{
\dimen0=#2\advance\dimen0 by #1
\leaders\hrule height \dimen0 depth -#1\hfill}
\titleformat{\section}{\large\scshape\raggedright}{}{0em}{}
\renewcommand\labelitemi{\raisebox{0.4ex}{\tiny$\bullet$}}
\renewcommand{\labelitemii}{$\cdot$}
\pagenumbering{gobble}
\begin{document}


\vspace*{56.28 mm}
\noindent Pursuing a \textbf{Minor} degree in \textbf{Artificial Intelligence} and \textbf{Data Science} from \textbf{C-MInDS, IIT Bombay}
\vspace{-10pt}
\noindent\section*{{\color{black!20!blue}\LARGE Scholastic Achievements\xfilll[0pt]{0.5pt}}}
\vspace{-10pt}

\begin{itemize}[itemsep = -0.65 mm, leftmargin=*]
    \item Achieved \textbf{99.81 Percentile} in {\bf JEE-Main} out of over 1 million candidates\hfill{\sl \small (2021)}
    \item  Secured {\bf All India Rank 1207} in {\bf JEE-Advanced} out of over 0.14 million candidates\hfill{\sl \small (2021)}
    \item Secured {\bf AP(Advanced Performer) grade} for excellent performance in \textbf{PH 108-Basics of Electricity \& Magnetism}, awarded to 27 out of over 1300 students taking the course\hfill{\sl \small (2022)}
    \item One of the \textbf{17 out of 1400+} students to secure a \textbf{Change of Branch} to the department of \textbf{Computer Science and Engineering} owing to excellent academic performance in first year at IIT Bombay\hfill{\sl \small (2022)}
    \item Secured {\bf All India Rank 275} in the prestigious \textbf{KVPY (Kishore Vaigyanik Protsahan Yojna)} SX and awarded fellowship by the Department of Sciences, \textbf{Indian Institute of Science(IISC) Bangalore}\hfill{\sl \small (2021)}


\end{itemize}
\vspace{-18pt}
\vspace{0pt}
\noindent\section*{\color{black!20!blue}\LARGE Key Projects\xfilll[0pt]{0.5pt}}
\vspace{-7pt}
\noindent\textbf{\large FastChat} \hfill{\sl \small (Autumn 2022)}\\
{\it Guide: Prof. Kavi Arya} $|$ {\it Ongoing Course Project : Software Systems Lab } \hfill{\it IIT Bombay}\\
\vspace{-15pt}
\begin{itemize}[itemsep = -0.65 mm, leftmargin=*]
    \item Developing a messaging software by building a network of clients interacting via servers acting as mediators
    \item Focusing on obtaining \textbf{high throughput} while using only \textbf{limited resources} dedicated for the servers
    \item Ensuring \textbf{low latency} of individual message deliveries and \textbf{end-to-end encryption} between clients
    \item Using \textbf{python socket library} to develop the network, using \textbf{open source libraries} for authentication and communication, \textbf{PostgreSQL} database to store the data and \textbf{bash} for scripting and collecting results
    \item Adding flair to this web application by implementing an interactive frontend using \textbf{HTML, CSS and JavaScript}
\end{itemize}
\vspace{\baselineskip}
\vspace{-10pt}
\noindent\textbf{\large Rail Planner}\hfill{\sl \small (Autumn 2022)}\\
{\it Guide: Prof. Supratik Chakraborty} $|$ {\it Course Project : Data Structures and Algorithms Lab} \hfill{\it IIT Bombay}
\\\vspace{-15pt}
\begin{itemize}[itemsep = -0.65 mm, leftmargin=*]
    \item Designed a simplified vesrion of a railway planner using various data structures and analyzed the space \& time complexity and the efficiency to demonstrate the \textbf{properties of different data structures in C++}
    \item Stored trains as a dictionary using \textbf{Hash Tables} and devised algorithms for fastest possible journies
    \item Used \textbf{BSTs and then AVL trees} for quick searching using the journey codes and used \textbf{Tries} to implement the autocompletion feature while searching for station names and added a feature to accept reviews for journies
    \item Used \textbf{Quicksort} to order trains by day and time, implemented the \textbf{KMP-string matching algorithm} for allowing review searches by using keywords and implemented \textbf{Heaps} to allow filtering the reviews by their rating
\end{itemize}
\vspace{\baselineskip}
\vspace{-10pt}
\noindent\textbf{\large Generating Representative Images from a Sample} \hfill{\sl \small (Autumn 2022)}\\
{\it Guide: Prof. Suyash Awate} $|$ {\it Ongoing Course Project : Data Analysis and Interpretation } \hfill{\it IIT Bombay}
\vspace{-3pt}
\begin{itemize}[itemsep = -0.65 mm, leftmargin=*]
    \item Used \textbf{MATLAB} to use a data set of images of various fruits and sampled random images to generate new
          representative fruit images using \textbf{Principal Component Analysis (PCA)}
    \item Used PCA to analyse images of handwritten digits from the \textbf{MNIST Database} and optimally reduce the
          dimensionality and reconstruct the image
    \item Implemented hyperplane fitting of 2 random variables and sampled points in the Euclidean Plane according to
          a given multivariate distribution
\end{itemize}
\vspace{\baselineskip}
\vspace{-10pt}
\noindent\textbf{\large Multiplayer Tic-Tac-Toe} \hfill{\sl \small (Autumn 2022)}\\
{\it Guide: Prof. Kavi Arya} $|$ {\it Course Project : Software Systems Lab } \hfill{\it IIT Bombay}\\
\vspace{-15pt}
\begin{itemize}[itemsep = -0.65 mm, leftmargin=*]
    \item Used \textbf{Java Socket Programming} for \textbf{inter process communication} using the \textbf{peer-to-peer model}
    \item Created the tic tac toe game using this model and handled various newtork and \textbf{IOStream exceptions}
\end{itemize}
\pagebreak
\vspace{\baselineskip}
\vspace{-10pt}
\noindent\textbf{\large Monte Carlo Analysis of Statistical Theorems} \hfill{\sl \small (Autumn 2022)}\\
{\it Guide: Prof. Suyash Awate} $|$ {\it Course Project : Data Analysis and Interpretation } \hfill{\it IIT Bombay}
\vspace{-3pt}
\begin{itemize}[itemsep = -0.65 mm, leftmargin=*]
    \item Used \textbf{MATLAB} to implement a Monte Carlo simulation of a given Probability distribution
    \item Plotted the probability and cumulative distribution functions of various distributions and empirically verified various statistical theorems such as the law of large numbers, Poison thinning and the
          Gaussian nature of the Random Walk
\end{itemize}
\vspace{\baselineskip}
\vspace{-12pt}
\noindent\textbf{\large Text File Editors}\hfill{\sl \small (Autumn 2022)}\\
{\it Guide: Prof. Kavi Arya} $|$ {\it Course Project : Software Systems Lab } \hfill{\it IIT Bombay}
\\\vspace{-15pt}
\begin{itemize}[itemsep = -0.65 mm, leftmargin=*]
    \item Developed an analog to the Linux Command Line utility \textbf{wc command} using the \textbf{awk programming language}
          that counts the number of characters, words and lines in a text file and also accepts flags similar to wc command
    \item Developed a program to check for valid email addresses using \textbf{sed} with pattern matching using \textbf{regular expressions}
    \item Implemented a \textbf{csv file editor} that formats columns based on customisable properties such as date, time and name
    \item Developed a program which changes the base of the number to a different given base using \textbf{bash scripting and awk}
    \item Developed a program to \textbf{encrypt} a piece of text when the words to encrypt and their corresponding cipher is given
\end{itemize}
\vspace{\baselineskip}
\vspace{-12pt}
\noindent\textbf{\large Personal Website}\hfill{\sl \small (Autumn 2022)}\\
{\it Guide: Prof. Kavi Arya} $|$ {\it Course Project : Software Systems Lab } \hfill{\it IIT Bombay}
\\\vspace{-15pt}
\begin{itemize}[itemsep = -0.65 mm, leftmargin=*]
    \item Made a personal website to be hosted on the CSE department server using \textbf{HTML and CSS}
    \item Added various advanced \textbf{CSS} features animations, transitions, static scroll images, modals, checkboxes and slideshows
    \item Used \textbf{JavaScript} to make the website interactive, gauge user-choices and render web-pages accordingly and deployed the website on an SSH server; used \textbf{BootStrap} to impelement standard navigation bars, footers and other features
\end{itemize}
\vspace{\baselineskip}
\vspace{-10pt}
\noindent\textbf{\large Bubble Trouble} \hfill{\sl \small (Autumn 2022)}\\
{\it Guide: Prof. Parag Chaudhuri} $|$ {\it Course Project : Computer Programming and Utilization } \hfill{\it IIT Bombay}
\vspace{-3pt}
\begin{itemize}[itemsep = -0.65 mm, leftmargin=*]
    \item Designed an interactive single player retro style game which impelements a bubble shooter to shoot random floating bubbles on the screen to demonstrate the \textbf{Object Oriented Paradigm in C++}
    \item Implemented event-handling using \textbf{XEvent} object extensively used the \textbf{C++ STL} and the Simplecpp library that was developed in-house by the institute to add the various features of the game
    \item Handled various events, assigning multiple responses by the game and designed the game for many levels of difficulty
\end{itemize}

\vspace{-17pt}
\noindent\section*{\color{black!20!blue}\LARGE Technical Skills\xfilll[0pt]{0.5pt}}
\vspace{-9pt}
\noindent\begin{tabular}{p{4.5cm} p{13.5cm}}
    \textbf{Programming Languages:} & C++, Python, MATLAB, Java, Bash, Solidity, Sed, AWK                   \\\\
    \textbf{Software \& Tools:}     & Git, \LaTeX{}, MySQL, NumPy, Pandas, Matplotlib, Doxygen, Sphinx, gdb \\\\
    \textbf{Web Development:}       & HTML, CSS, JavaScript, BootStrap                                      \\
    %\textbf{Software} & Git, \LaTeX, AutoCAD
    %\textbf{Web Development} & Django, HTML, CSS, JavaScript, PHP, Bootstrap, Jquery, Android Studio  \\
    %   \textbf{Extra Courses} & Probability theory, Advance Network Security and Cryptography, Derivative Pricing
\end{tabular}
\vspace{-12pt}

%\vspace{-17pt}

\noindent\section*{\color{black!20!blue}\LARGE Courses Undertaken\xfilll[0pt]{0.5pt}}
\vspace{-8pt}
% \begin{itemize}[itemsep = -0.75 mm, leftmargin=*]
%   \item {\bf Mathematics:} Calculus, Linear Algebra
%   \item {\bf Computer Science:} Computer Programming and Utilisation, Abstractions for Programming, Data Structures and Algorithms*, Data Analysis and Interpretation*, Software Systems Lab*, Discrete Structures*, Computer Networks**, Digital Logic Design**, Design and Analysis of Algorithms**, Logic for Computer Science**
%   \item {\bf Misc:} Quantum Physics, Introduction to Electronic Circuits*, Organic Chemistry, Optimisation Models*
% \end{itemize}
\noindent\begin{tabular}{m{36mm} m{13.2cm}}
    \textbf{Mathematics}      & Calculus, Linear Algebra, Differential Equations                                                                                                                                                                                              \\\\
    \textbf{Computer Science} & Computer Programming and Utilization, Discrete Structures*, Data Structures and Algorithms*$^{\#}$, Data Analysis and Interpretation*, Software Systems Laboratory*                                                                           \\\\
    \textbf{Miscellaneous}    & Optimization Models*, Introduction to Electric and Electronic Circuits*, Quantum Physics and Application, Basics of Electricity and Magnetism, Engineering Graphics and Drawing, Organic and Inorganic Chemistry, Physical Chemistry, Biology \\                                                                                                                                                                                                                                               \\
\end{tabular}
\vspace{-10pt}
\begin{flushright}
    \sl \small  (* to be completed by November 2021)\\
    \sl \small (\# Theory + Lab)
\end{flushright}
\vspace{-15pt}
\noindent\section*{\color{black!20!blue}\LARGE Extracurricular\xfilll[0pt]{0.5pt}}
\vspace{-7pt}
\begin{itemize}[itemsep = -0.65 mm, leftmargin=*]
    \item Successfully completed one year under \textbf{National Sports Organization(NSO)} in \textbf{Chess} at IIT Bombay\hfill{\sl \small (2022)}
    \item Pitched a \textbf{Business Model Canvas} for a startup in the health sector for the EnB Buzz competition conducted by the \textbf{Entrepreneurship cell of IIT Bombay}\hfill{\sl \small (2021)}
    \item Participated in \textbf{Google Hashcode 2021}\hfill{\sl \small (2021)}
    \item Worked in a team of 4 to make an ESP32 \textbf{WiFi-controlled bot} for XLR8 conducted by \textbf{ERC, IITB}\hfill{\sl \small (2022)}

\end{itemize}
\end{document}


\item Studied various probability and cumulative density functions of various distributions like Binomial, Poisson, Gaussian, Laplacian etc. and computed their mean and variance analytically and calculated deviations from theoretical data
\item Generated instances of random walker simulations and analyzed their trajectories graphically along with determining the form of the histograms of their final locations for varying number of walkers and steps taken
\item Empirically verified various statistical theorems such as the law of large numbers, Poison thinning and Gaussian nature of the random walk by running appropriate random simulations using MATLAB and Python numpy, matplotlib
\item Composed random variables by making appropriate functions to model a draw from the desired transformed PDFs
\item Participated in a team of 3 and wrote a working script and successful submission in \textbf{Google Hashcode 2021}\hfill{\sl \small (2021)}